% DO NOT EDIT - automatically generated from metadata.yaml

\def \codeURL{https://github.com/theoboraud/JaegerESN}
\def \codeDOI{}
\def \dataURL{}
\def \dataDOI{}
\def \editorNAME{}
\def \editorORCID{}
\def \reviewerINAME{}
\def \reviewerIORCID{}
\def \reviewerIINAME{}
\def \reviewerIIORCID{}
\def \dateRECEIVED{}
\def \dateACCEPTED{}
\def \datePUBLISHED{}
\def \articleTITLE{[Re] A Neurodynamical Model for Working Memory}
\def \articleTYPE{Replication}
\def \articleDOMAIN{Computational Neuroscience}
\def \articleBIBLIOGRAPHY{bibliography.bib}
\def \articleYEAR{2019}
\def \reviewURL{}
\def \articleABSTRACT{Neurodynamical models of working memory (WM) should provide mechanisms for storing, maintaining, retrieving, and deleting information. Raznan Pascanu and Herbert Jaeger have suggested that memory states correspond, intuitively in terms of nonlinear dynamics, to attractors in an input-driven system, giving a simple WM model training all performance modes into a Recurrent Neural Network (RNN) of the Echo State Network (ESN) type. In this replication, we reproduced this WM model in Python in order to see wether or not these observations could be observed on a slightly different model made from scratch.}
\def \replicationCITE{}
\def \replicationBIB{}
\def \replicationURL{}
\def \replicationDOI{}
\def \contactNAME{Theophile Boraud}
\def \contactEMAIL{theoboraud@outlook.com}
\def \articleKEYWORDS{Recurrent Neural Networks, Echo State Networks, Working Memory}
\def \journalNAME{ReScience C}
\def \journalVOLUME{5}
\def \journalISSUE{1}
\def \articleNUMBER{}
\def \articleDOI{}
\def \authorsFULL{Theophile Boraud and Anthony Strock}
\def \authorsABBRV{T. Boraud and A. Strock}
\def \authorsSHORT{Boraud and Strock}
\title{\articleTITLE}
\date{}
\author[1,4,\orcid{0000-0003-0388-0392}]{Theophile Boraud}
\author[1,2,3,\orcid{0000-0002-6725-561X}]{Anthony Strock}
\affil[1]{INRIA Bordeaux Sud-Ouest, Bordeaux, France}
\affil[2]{LaBRI, Université de Bordeaux, Institut Polytechnique de Bordeaux, Centre National de la Recherche Scientifique, UMR 5800, Talence, France}
\affil[3]{Institut des Maladies Neurodégénératives, Université  de Bordeaux, Centre National de la Recherche Scientifique, UMR 5293, Bordeaux, France}
\affil[4]{Department of Computer Science, University of Warwick, Coventry, United Kingdoms}
